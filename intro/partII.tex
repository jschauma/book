\begin{quote}
{\em A concept is stronger than a fact.}
-- Charlotte Perkins Gilman\index[names]{Perkins Gilman, Charlotte}
\end{quote}

Having introduced the field of System Administration
in the previous section, we will now focus on a number
of important technologies and principles.  System
Administrators need to approach their job
holistically; as discussed, a ``system'' is comprised
of many different interdependent components and the
maintenance of the complex whole requires an intricate
understanding of each.

In this section, we will look at the (computer) system
from the ground up; the chapters have a certain order
and the materials in one chapter tend to build upon
topics covered in those preceding it.  However, as
both an instructor or student, you are encouraged to
work your way through this book in any order you like.
To help you better find the topic that is right for
your progress and pace, let us briefly summarize the
chapters in this part. \\

To begin our presentation of fundamental concepts and
technologies, we start with an overview of storage
models, discuss the benefits, drawbacks, and
properties of local storage devices, network storage,
and even further abstracted cloud storage\index{Cloud
Storage} models.  As if building a new machine, we
start out at a fairly low level by understanding the
physical disk structure and partitions.  Next, we
review the traditional \gls{ufs}\index{Unix File
System} in some detail in order to illustrate general
file system concepts.

Next, in Chapter \ref{chap:software-installation}, we
cover software installation concepts.  We divide the
overall chapter into different sections, focusing on
firmware, BIOS and operating system installation,
system software, and the concept of the basic
operating system (versus a kernel all by itself), and
finally third party applications.  We present
different package management solutions, including
binary- and source-based approaches and discuss patch
management, softwarr upgrades, and security audits.

Once we understand how software is installed and
maintained on our systems, we will spend some time
examining how to configure our systems for their
different tasks and how to use centralized systems
to ensure consistency across multiple and diverse
environments.  In Chapter
\ref{chap:configuration-management}, we will review
the fundamental requirements a configuration
management\index{Configuration Management} system has
to meet, discuss architectural decisions and pay
particular attention to their impact on scalability
and security.  This chapter touches upon a number of
interesting aspects including, but not limited to,
user management, access control, role definition,
local or system-wide customizations, and eventually
hints at what has become known as ``Service
Orchestration\index{Service Orchestration}'', a
concept we will revisit later in Part
\ref{part:services} in Chapter \ref{chap:services}.

Having seen the power of automating system and
software deployment and having faced the vast amounts
of data collected and processed, we will take a step
back and review in detail the basic concepts of
automating administrative tasks.  In Chapters
\ref{chap:automation} and
\ref{chap:building-scalable-tools}, we dive into the
how, when and why of automation in general.  Once
again, we will build fundamental knowledge and
understanding by reviewing {\em concepts} rather than
explicit code excerpts.  We will differentiate between
``scripting'', ``programming'' and ``software
engineering'', and focus on the art of writing {\em
simple} tools for use in a System Administrator's
daily life.  In doing so, we will once again revisit a
few topics from earlier in the book and deepen our
understanding of (software) documentation and package
management.

Following this, we discuss networking in Chapter
\ref{chap:networking}.  This chapter can be viewed as
being entirely ``out of order'', as virtually all
previous topics in a way require or relate to working
with a network.  Within the context of System
Administration, we being our coverage of this
significant topic by traversing the layers of the OSI
stack.  We include practical and detailed examples of
how to analyze network traffic and how packets are
transferred from one application to another on the
other side of the internet.  The discussions will
cover both IPv4 and IPv6 equally and include the
implications and possible caveats of networking with a
dual stack.

The internet architecture and some governing standards
bodies will also be covered here.  Programs covering
these topics in depth in pre-requisites may consider
skipping this chapter, even though we believe to
approach it from a unique angle, i.e. the System
Administrator's point of view.

The last chapter in Part \ref{part:fundamentals}
(Chapter \ref{chap:security}) is entitled ``System
Security''.  Like the previous chapter, it, too, feels
a bit out of order: all previous chapters include
specific security related notes and explicitly
identify security concerns in any technology
discussed, yet this chapter finally takes a step back
and discusses security not from an application point
of view, but from a general all-encompassing view.
That is, we discuss the basic concepts of Risk
Assessment and Risk Management, the different threat
scenarios one might experience and how to best respond
to them.  We will cover basics of encryption and how
it provides different layers of security via assurance
of confidentiality, integrity and authenticity. A
second particular focus will be on balancing usability
with security as well as the social implications of
any instated security policy.  \\

Having built a foundation of core concepts and
technologies, we then enter Part \ref{part:services},
where we discuss management of complex services by
building upon the previous chapters.  We will discuss
different service architectures (e.g. monolithic vs.
microservices\index{microservices}), complexity
implications and considerations, service orchestration
and maintenance in Chapter \ref{chap:services};
revisiting certain aspects of different file systems
and storage devices, we cover the concepts relating to
backups and disaster recovery in Chapter
\ref{chap:backup}.  We distinguish between backups for
different use cases (prevent temporary data loss,
long-term archival of data, file/file system
integrity) and help students learn to develop an
appropriate disaster recovery plan.

We will analyze the need for large scale system and
event logging, which then ties directly into the area
of system and network monitoring in Chapter
\ref{chap:monitoring}.  Here we will discuss the
\gls{snmp}\index{SNMP} as well as a few industry
standard tools built on this protocol and review how
they can be used to measure metrics such as system
response time, service availability, uptime,
performance and throughput on an enterprise scale.
\\

We will conclude our whirlwind tour across all the
diverse areas of System Administration in Part
\ref{part:meta}, hinting at the fact that we really
only have barely scratched the surface in many ways.
We will outline major industry trends and developments
that we did not have the time or space to include here
in Chapter \ref{chap:missing}, before circling back to
the definition of our profession in Chapters
\ref{chap:ethics} and \ref{chap:future}, where we
elaborate on the legal and ethical obligations and
considerations before we take a brief look at what
might lie ahead. \\

As you can tell, each topic is far reaching, and we
cannot possibly cover them all in every possible
detail.  For this reason, we focus not on specific
examples but on the basic principles.  Instructors
should try to choose real-world examples from personal
experience to illustrate some of these concepts in
class, as we will present some case studies where
appropriate.  Students, on the other hand, are
encouraged to relate the topic to their own
experiences and to deepen research based on their
interests.
