\chapter{Documentation Techniques}
\label{chap:documentation}

\begin{quote}
{\em Lack of documentation is becoming a problem for acceptance.}
-- Wietse Venema\index[names]{Venema, Wietse}
\end{quote}

\begin{figure}[hb]
	\raggedleft
	\includegraphics[width=.15\textwidth]{03/pics/pen}
	\label{fig:pen}
\end{figure}

%\begin{figure}[b]
%	\captionsetup{justification=justified,singlelinecheck=false}
%	\caption[A pen]{Mightier than the sword...
%		\label{fig:pen}}
%\end{figure}

We noted in Chapter \ref{unix:basics:manual} that one
of the many ways in which the Unix operating system
distinguished itself from other systems was that it
included extensive documentation of high quality.
Each tool provided by the OS came with a {\em manual
page}\index{Manual Pages} describing its use, each
library included a description of its interfaces, and
configuration files were accompanied by documentation
elaborating on its syntax and format.

Even simple systems cannot be operated, maintained,
updated -- or in a word, administered -- without an
abstract description of the various moving parts.

But while every System Administrator values good
documentation, and will happily agree that, yes, all
systems should in fact be properly documented,
runbooks created, procedures described, and additional
information referenced, it does take practice and
conscious dedication to develop a habit of writing
high quality system documentation\index{system
documentation}.

For this reason, and before we dive into the
fundamental technologies and concepts in Part
\ref{part:fundamentals}, let us briefly focus on a few
principles that help us provide better documentation
due to a better understanding of what purpose it
serves.

\section{System Documentation Writing 101}
\label{documentation:system-documentation-writing}

Writing is communication.  You wish to provide
information for your organization, your users, or
colleagues.  As a reader of system documentation, you
are in search of information.  This exchange of
knowledge is direct communication between the author
of the documents and the reader.

The motivation, purpose, and effective techniques
used in any kind of writing are similar, so allow us
to begin with some advice that, while it may sound like
it was cribbed from a ``Creative Writing 101''
syllabus, still remains sound.

\subsection{Know Your Audience}
\label{documentation:creative:audience}

Understanding who may be reading our documentation is
essential to providing the right information, just as
a software engineer must understand who the product's
users will be.

Knowing your audience means that you understand what
information they are looking for, what background
knowledge they may have, what additional documentation
or resources they may have access to, and so on.

When we create system documentation, we usually target
the following readers\footnote{You will note that this
same advice for ``knowing your audience'' translates
near verbatim to knowing your {\em users} when writing
software.  We will revisit this concept in chapter
\ref{chap:building-scalable-tools}.}: \\

{\em We write documentation for ourselves.} This is
the simplest case, as we know our target audience well
-- or so we think!  For the most part, we are writing
things down so we can repeat them later on, so we
don't forget, so we have a point of reference in the
future.  This means that we have a pretty good idea
what the prospective reader might expect to find, what
they might know about the infrastructure etc.

{\em We write documentation for other system
administrators.} This case is still fairly straight
forward.  We expect our target audience to be very
technical and have a very good understanding of our
systems.  For example, we write down the steps of a
procedure that our colleagues may not normally
perform, or we document the setup of a system
component we are in charge of that is crucial to the
overall operations of our organizations so that they
may use this document as a reference when we are not
available.

{\em We write documentation for other technical
people.} Our services are used by a number of people
in our organization.  Some of these are very technical
expert users, software developers or other engineers,
perhaps.  In order to allow these people to get
information about our systems quickly, we provide
end-user documentation which allows more efficient use
of the resources we make available.

{\em We write documentation for our users.} The
systems we maintain provide a certain service and thus
has users, some of whom may be internal to our
organization and others which may be outside
customers.  It is not rare that the people in charge
of maintaining the systems provide documentation to
these end-users in one way or another to allow them to
utilize the service or help them find help when things
break down.

{\em We write documentation for other people
everywhere.} System administrators rely on the
Internet at large to find answers to the rather odd
questions they come up with in the course of their
normal work day.  Frequently we encounter problems
and, more importantly, find solutions to such problems
that are of interest to other people, and so we strive
to share our findings, our analyses, and our answers.

Some of the most interesting technical blogs are
written by system administrators or operational teams
and describe how they scale their infrastructure to
meet their demands.  Likewise, system administrators
present their findings and solutions at industry
conferences, write technical papers or participate in
Internet standards development, all of which require
expert documentation and writing skills.  \\

The differences in target audience have a certain
impact on how you would structure your documentation
as well as, in some cases, implications on access
control of the information in question.  Internal
documentation targeted towards your peers may well
include certain privileged information that you would
rather not have made available to the entire
organization (or outsiders via the Internet).

As much as the target audience differs, though, the
best advice for writing good technical documentation
is to avoid making any assumptions about the reader's
prior knowledge and familiarity with the systems in
questions.  At the same time, we need to carefully
identify and reference whatever assumptions,
unavoidable and necessary as they may be, {\em are}
being made.

Secondly, it is important to clearly distinguish {\em
internal} from {\em external} documentation.  The
distinction here is not so much between your corporate
or legal identity and the public at large -- an easy
and obvious one to draw and enforce -- but between
distinct entities within a single organization.
Documentation containing certain details regarding
your infrastructure's network architecture, access
control lists, location of credentials and the like
may simply not be suitable to be made available
outside your department.  Unfortunately the boundaries
can be less than obvious as people move in and out of
departments as an organization evolves.

\section{Online Documentation}
\label{documentation:online-documentation}

One of the main differences from creative writing is
that our documentation is in almost all cases intended
to be read online.  Even though certain documents such
as a contact information registry of the support staff
or the emergency escalation procedures may be useful
to have in print, virtually all information we collect
and make available are read in an online or computer
context.  This has a profound impact on the structure
and style of our documents.

Studies (\cite{doc:online-literacy},
\cite{doc:nielson-web-writing}, et al.) have shown
that not all ``reading'' is equal:  content consumed
via a web browser, for example, is frequently only
glanced through, skimmed over or otherwise not paid
full attention to.  When we read documents online, we
tend to search for very specific information; we
attempt to extract the ``important'' parts quickly and
tend to ignore the rest.

As a result, online documentation needs to be much
more succinct than information intended for actual
print.\footnote{Imagine reading this book entirely
online, perhaps as a series of blog posts.  You would
likely skip a lot of content and perhaps rely on
summaries, bullet points and other visual emphases to
find what might seem worth reading.  I certainly would
have structured the content differently.}  Sentences
and paragraphs need to be short and simple; the
subject matter covered concise.  Additional
information, further references, or more in-depth
analyses can always be made available to the reader
via appropriate linking.  Section headings, bold or
italic fonts and the use of itemized lists make it
much easier for the reader to quickly absorb the
content.

In addition, we are much less inclined to tolerate
superfluous information when reading online.  While
Dale Carnegie's\index[names]{Carnegie, Dale} advice
(``Tell the audience what you're going to say, say it;
then tell them what you've said.'') is sound for
effective presentations or traditional writing,
building up each document you create with an
introduction and summary incurs too much repetition to
be useful.  Getting straight to the heart of the
matter is usually advisable in our context.

Another benefit of providing documentation online is
that it becomes immediately searchable and can be
linked to other sources of information.  Helping your
readers to find the content they are looking for
becomes trivial if your documentation framework
includes automated index generation and full text
search.  To make it easier for your readers, be sure
to use the right keywords in the text or use ``tags''
-- remember to include or spell out common acronyms or
abbreviations depending on what you and your
colleagues or users most commonly use!

\begin{sidenote}
{\bf tl;dr} \\
Sometime in 2011, an acronym established itself in the
``blogosphere'': {\em tl;dr -- too long; didn't read}.
This acronym (in general use since at least 2003)  was
frequently accompanied by a short summary of the
content in question, thus allowing online readers
lacking the patience to actually read the document to
nevertheless form opinions about it.  While it is
certainly useful to provide a succinct summary of the
content in e.g. an email (where an appropriate
'Subject' line or an introductory sentence may
suffice), ii you feel inclined to provide a {\em
tl;dr\index{tl;dr}} summary to your documentation,
chances are that you're addressing the wrong
audience.
\end{sidenote}


\section{Different Document Types}
\label{documentation:types}

Knowing your audience and understanding what you wish
to communicate to them are the most important aspects
to understand before starting to write documentation.
From these two main points derive a number of
subsequent decisions that ultimately influence not
only the process flow but also the structure of the
document.  Drawing parallels to software engineering
once more, where we decide on the programming language
or library to use based on its suitability to actually
solve the given problem, we find that here, too, we
need to use the right tool for the job.  We cannot
structure our end-user manual as a checklist, nor can
we provide online help or reference documentation as a
formal paper.

Each document type calls for a specific writing style,
a unique structure and flow of information.  In this
section, we will take a look at some of the most
common distinct categories of system documentation.

\subsection{Processes and Procedures}
\label{documentation:types:processes}

\vspace{.25in}
\begin{tabular}[ht]{ l p{.75\textwidth}}
	\hline
	{\bf Purpose} & describe in detail how to perform a specific task; \newline
			document the steps to follow to achieve a specific goal; \newline
			illustrate site-specific steps of a common procedure \\
	\hline
	{\bf Target audience} & all system administrators in the organization \\
	\hline
	{\bf Structure} & simple, consecutive steps; \newline
				checklist; \newline
				bullet points with terse additional information\\
	\hline
\end{tabular}
\vspace{.25in}

This is probably the most common type of document you
will have in your information repository.  Without any
further classification, pretty much anything you
document will fall into this category; it is a broad
definition of what most people think of when they
think about system documentation.  It is worth taking
a closer look, as different types of process
documentation exist and hence call for different
formats or presentation.

{\em Processes} usually provide a description of what
is done in a given situation, what the correct or
suitable steps are to resolve a given problem or how a
routine task is accomplished.  These documents are
generally focused on the practical side of the
business or organization.

{\em Procedures} are focused slightly more on the
operational side of things.  They not so much describe
what is to be done but list the actual commands to
issue to yield a specific result.  It is not uncommon
to have well-documented procedures be turned into a
``runbook\index{runbook}'', a list of ordered steps to
follow and commands to issue under specific
circumstances.  Carefully and correctly written, a
runbook can frequently be used to let even
inexperienced support staff respond to an emergency
situation.  The procedures listed usually include
simple if-this-then-that directions with exit points
to escalate the problem under certain circumstances.
Furthermore, a good runbook usually makes for an
excellent candidate for automation -- more on that in
Chapter \ref{chap:automation}.

Another type of document can be considered a
subcategory of a process document: the ubiquitous {\em
HOWTO\index{HOWTO}}.  Instructional descriptions of
how to accomplish a certain task generally skip any
reasoning or justification of the ultimate goal and
focus on the individual steps to reach it instead.
Using simple sentences, itemized, or enumerated lists
and often times including sample invocations and
command output, this document will outline the details
of each task, noting in particular how the environment
in question may differ from those covered by other
such guides.

When writing process and procedure documents, it is
usually a good idea to include actual command
invocations together with their complete output rather
than to fabricate seemingly illustrative invocations.
By using the exact commands that are applicable in the
given environment we make it easy for the reader to
follow the process as well as to find this document if
they only remember a specific command related to the
task at hand.


\subsection{Policies}
\label{documentation:types:policies}

\vspace{.25in}
\begin{tabular}[ht]{ l p{.75\textwidth}}
	\hline
	{\bf Purpose} & establish standards surrounding the use of available resources \\
	\hline
	{\bf Target audience} & all users of the given systems \\
	\hline
	{\bf Structure} & full text, including description and rationale \\
	\hline
\end{tabular}
\vspace{.25in}

Most people think of {\em policies} as lengthy
documents full of footnotes and fine print, frequently
drafted by your organization's legal or human
resources departments.  But there are a large number
of necessary policies that are -- or should be --
written by (or with the help of) System
Administrators, as they relate to and impact their
work directly.  In particular, computer related
protocols such as the ``Terms of Service\index{Terms
of Service}'' or Acceptable Use
Policies\index{Acceptable Use Policies} (of the
systems, not the software product(s) possibly offered
to outside customers on said systems), various
\glslink{sla}{Service Level Agreements}\index{Service
Level Agreement} (SLAs), and information security
guidelines represent an abstract description of
practical implementations on the system side under the
given circumstances.

Unfortunately, it is not uncommon to have such
guidelines be drafted completely outside the
functional or possible environment in which they need
to be turned into action.  In such cases, these
policies become meaningless, as they are adhered to
only in the letter, but not in the spirit -- the
\gls{pcidss}\index{PCI-DSS} compliance is, sadly, a
common example.

The documents governing the interactions with the
customers or consumers must be specific, clear, and
avoid loopholes.  Hence, they should be written with
the direct feedback and input from the people in
charge of implementing the rules and regulations
prescribed.

For an excellent and much more in-depth discussion of
Computing Policy Documents\index{Computing Policy
Documents}, please see \cite{doc:lisa-policies}.

\subsection{Online Help and Reference}
\label{documentation:types:help}

\vspace{.25in}
\begin{tabular}[ht]{ l p{.75\textwidth}}
	\hline
	{\bf Purpose} & list and index available resources; \newline
			illustrate common tasks; \newline
			describe common problems and provide solutions \\
	\hline
	{\bf Target audience} & all users of the given systems; \newline
				possibly restricted set of privileged users (depending
				on the resources indexed) \\
	\hline
	{\bf Structure} &  simple catalog or itemization; \newline
				short question-and-answer layout; \newline
				simple sentences with example invocations \\
	\hline
\end{tabular}
\vspace{.25in}

In this category you will find all the various
documents provided by System Administrators, IT
Support, Help Desk\index{Help Desk}, and similar
groups that are intended to allow users to find
solutions to their problems without having to engage
personal assistance.  A common solution includes
increasingly complex documents entitled
``FAQ\index{FAQ}'' (for ``Frequently Asked
Questions'').

When creating an FAQ compilation, it is important to
focus on the questions your users {\em actually} ask,
not the ones that you {\em think} they might ask.  All
too often these documents include answers to questions
that are not, in fact, all that frequent, but that the
author happened to have an answer to or would like the
users to ask.

It should be noted that it may well be necessary to
separate some of these documents for one audience from
those provided to another.  The references you provide
to your peers may include privileged information, and
the style in which you guide inexperienced users would
be ill-suited to help your expert users find solutions
to their problems.  Once again, knowing your audience
is key.

In addition, it is important to distinguish between
{\em reference} material and what might be called a
``User Guide''.  The latter is conceptual in nature;
it might explain the nature of the problem and how it
is solved.  The former is terse, simple, and strictly
limited to the available options; it does not
elaborate on why things are the way they are, it
merely presents the interface.


\subsection{Infrastructure Architecture and Design}
\label{documentation:types:infra-design}

\vspace{.25in}
\begin{tabular}[ht]{ l p{.75\textwidth}}
	\hline
	{\bf Purpose} & describe in great detail the design of the infrastructure; \newline
			illustrate and document information flow; \newline
			{\em document reality} \\
	\hline
	{\bf Target audience} & other system administrators in the organization \\
	\hline
	{\bf Structure} &  descriptive sentences with detailed diagrams; \newline
				references to rationale and decision
				making process behind the designs \\
	\hline
\end{tabular}
\vspace{.25in}

Among the most important sets in a System
Administrator's information repository are the
infrastructure architecture and design documents.  In
these, the system architecture is described in great
detail; they aim to provide an accurate representation
of reality together with references to the rationale
behind certain decisions.\footnote{At times, this
``reality'' may, in fact, be a {\em desired} or {\em
ideal} reality; it is important to explicitly note
wherever this diverges from the actual architecture as
implemented.}

These records are meant to be authoritative and can be
relied on to resolve disputes about what traffic may
flow between security zones, to make decisions about
where to place a new device, or to identify
bottlenecks in throughput, to name just a few
examples.  Although the quality and accuracy of these
documents ought to be a priority, all too often
descriptions of the architecture or network are
neglected to be updated, quietly drifting into
obsolescence.

One of the reasons for this perpetual state of being
outdated is the inherent complexity of the subject
matter.  Complex infrastructures are frequently
described in complex, or worse, convoluted ways.  Keep
your content simple and to the point: you should be
able to describe your network in a few straightforward
sentences.  If you feel yourself in need of numerous
illustrations or lengthy explanations, break out
larger components into less complex blocks,
each documented individually.

\subsection{Program Specification and Software Documentation}
\label{documentation:types:program-specification}

\vspace{.25in}
\begin{tabular}[ht]{ l p{.75\textwidth}}
	\hline
	{\bf Purpose} & describe a single software tool, its capabilities and limitations; \newline
			illustrate common usage; \newline
			provide pointers to additional information \\
	\hline
	{\bf Target audience} & all users of the tool, inside or outside of the organization \\
	\hline
	{\bf Structure} &  short, simple, descriptive sentences; \newline
				example invocations including sample output; \newline
				possibly extended rationale and detailed
				description, command or syntax reference
				etc. via in-depth guide \\
	\hline
\end{tabular}
\vspace{.25in}

One of the mistakes System Administrators tend to make
is to not treat their tools as a complete software
product.  Software packages installed from third
parties are often ridiculed or derided for having
incomplete documentation or lacking examples, yet at
the same time, our own little tools and shell scripts
remain stashed away in our {\tt \~{}/bin} directory,
possibly shared with colleagues.  We will cover this
topic in much more detail in Chapter
\ref{chap:building-scalable-tools}, but let us
consider how our expectations of other people's
software differ from what we provide ourselves.

Every software package, every tool, every script,
needs documentation.  In addition to the generic
information we might find on upstream vendors'
websites such as where to download the software from,
what users are in need of is practical examples.  With
the tools you have written yourself, you have the
distinct advantage of knowing {\em precisely} the use
cases most relevant to the users of the software, and
so you should provide them.

I have made it a habit of creating a very simple
``homepage'' for every software tool I write.  On it,
I include a short summary of what the tool was
designed to do, where in the revision control
repository the sources can be found, where feature
requests and bug reports can be filed, how the
software can be installed, a pointer to the change
log, and version history and last and certainly not
least a number of example invocations with expected
output.  This information is also included in the
project's repository in the form of a {\tt README}.

If the software in question becomes more complex, you
may wish to reference additional documents, including
pointers to the software architecture, the rationale
for certain design decisions, a software specific FAQ
etc.

\section{Formats and Collaboration}
\label{documentation:formats-and-collaboration}

Unlike other kinds of writing, creating and
maintaining system documentation is not a solitary
task.  Instead, you collaborate with your colleagues
to produce the most accurate information possible, and
allowing end users to update at least some of the
documents you provide has proven a good way to keep
them engaged and improve the quality of your
information repository.

Depending on the type of document, you may wish to
choose a different method of enabling collaboration.
Some documents should be treated as less flexible or
mutable than others: for example, you probably do not
want your customers be able to modify the SLAs you
agreed to, but you {\em do} want to encourage your
users to contribute use cases or corrections to your
software documentation.

At the same time, you also want to be careful to not
prevent others from contribution to your documentation
by choosing a document format that raises the barrier
to collaboration.  While images and diagrams may be
useful, text ought to be your primary content: it can
can easily be skimmed, searched, glanced through in a
matter of seconds; a screencast or video, to cite an
extreme example, must be watched one excruciating
minute at a time (not to mention the challenges
non-textual media pose to visually impaired users).

Whichever collaborative documentation solution you
choose -- whichever ``Wiki\index{Wiki}'' or repository
or markup dialect -- remember that the easier you make
it for yourself, your colleagues and your users to
create content, to keep it up to date and to correct
even minor errors, the better your documentation will
be.  And with good documentation come happy users,
fewer alerts, and significantly less stress for you...

\begin{quote}
{\em Incorrect documentation is often worse than no documentation.} --
Bertrand Meyer\index[names]{Meyer, Bertrand}
\end{quote}


\vfill
\pagebreak

\chapter*{Problems and Exercises}
\addcontentsline{toc}{chapter}{Problems and Exercises}
\section*{Problems}

\begin{enumerate}
\item
Identify a tool or utility you use on a regular basis which does not have
an adequate manual page.  Write the fine manual!  Submit it to the tool's
author/maintainer.

\item
Identify existing systems documentation in your environment.  What is the
documents' intended purpose, and do they meet that goal?  What format are
they in?  Are they up to date and accurate?  How and when are changes
made?

\item
Many Open Source projects also are transparent in the ways that their
infrastructure is maintained and operated -- identify a major project and
review their documentation.  Compare to how different companies present
their best practices (for example on a company blog, in presentations at
conferences, in articles about their infrastructure, ...).  Does
documentation play a significant role?

\end{enumerate}

\pagebreak

\bibliographystyle{plainnat}
\begin{thebibliography}{99}

\bibitem{doc:limoncelli-tmfsa}Thomas A. Limoncelli, {\em Time Management for
System Administrators}, O'Reilly Media, 2005

\bibitem{doc:lisa-policies}Barbara L. Dijker (Editor), {\em Short Topics in System
Administration: A Guide To Developing Computing Policy Documents},
USENIX Association, Berkeley, CA, 1996

\bibitem{doc:strunk-white}William Strunk Jr., E. B. White, {\em The Elements
of Style}, Longman, 4th Edition, 1999 (The original text is now in the
public domain and available online, for example at
{\url http://www.gutenberg.org/ebooks/37134} (visited January 23, 2017).)

\bibitem{doc:handbook-of-technical-writing}Gerald J. Alred, Charles T. Brusaw,
Walter E. Oliu, {\em The Handbook of Technical Writing}, St. Martin's
Press, 8th edition, 2006

\bibitem{doc:online-literacy}Mark Bauerlein, {\em Online Literacy Is a
Lesser Kind}, The Chronicle of Higher Education, September 19, 2008; also
available on the Internet at
{\url http://greatlibrarynews.blogspot.com/2008/09/online-literacy-is-lesser-kind.html}
(visited January 23, 2017)

\bibitem{doc:nielson-web-writing}Jakob Nielsen, {\em Writing for the Web},
miscellaneous papers and links; on the Internet at
{\url http://www.useit.com/papers/webwriting/} (visited April 06, 2012)

\bibitem{doc:schaumann-art-of-plain-text}Jan
Schaumann, {\em The Art of Plain Text}; on the
Internet at
{\url https://www.netmeister.org/blog/the-art-of-plain-text.html}
(visited January 28, 2016)

%http://www.perlmonks.org/?node_id=130249
%http://www.codinghorror.com/blog/2006/08/how-to-write-technical-documentation.html
%http://jacobian.org/writing/great-documentation/
%https://www.readwriteweb.com/start/2010/08/tips-for-writing-good-document.php
%http://itmanagersinbox.com/1556/how-to-write-it-technical-documentation/
%http://www.suite101.com/lesson.cfm/16712/257
%http://www.agilemodeling.com/essays/agileDocumentationBestPractices.htm
%https://jeffspost.wordpress.com/2007/11/18/twelve-lessons-from-writing-documentation/

\end{thebibliography}
